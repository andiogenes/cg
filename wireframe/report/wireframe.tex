% Использован шаблон:
% https://www.writelatex.com/coursera/latex/1.1
% http://coursera.org/course/latex


\documentclass[a4paper,12pt]{article}

\usepackage{cmap}
\usepackage[T2A]{fontenc}
\usepackage[utf8]{inputenc}
\usepackage[english,russian]{babel}
\usepackage{fancyhdr}
\usepackage{listings}


\pagestyle{fancy}
\fancyhf{}
\lhead{Антон Завьялов, ПИ-72}
\rhead{\textbf{Лабораторная №1. Вариант 7}}
\cfoot{\thepage}

\makeatletter
\def\@seccntformat#1{%
  \expandafter\ifx\csname c@#1\endcsname\c@section\else
  \csname the#1\endcsname\quad
  \fi}
\makeatother

\begin{document} % Конец преамбулы, начало текста.

\begin{center}
  \textbf{Лабораторная работа №1 по дисциплине\linebreak"Компьютерная графика"\linebreak\linebreakВыполнил студент группы ПИ-72 Завьялов А.А.}\\
\end{center}

\section{\normalsize{Задание}}
\begin{flushleft}
  Разработать программу аффинных преобразований и проецирования 3D-проволочного объекта. Интерфейс должен позволять управлять текущим преобразованием объекта мышью или клавиатурой.
\end{flushleft}

\begin{flushleft}
  Реализовать все элементарные преобразования. Кроме того, реализовать дополнительное динамическое преобразование (анимацию) по варианту.
  \linebreak\linebreak
  \textbf{Вариант 7.} Вращение относительно геометрического центра объекта со случайной сменой направления (смена направления должна осуществляться плавно!)
\end{flushleft}

\section{\normalsize{Ход работы}}
\begin{flushleft}
  
\end{flushleft}

\section{\normalsize{Исходный код}}

\end{document}

