% Использован шаблон:
% https://www.writelatex.com/coursera/latex/1.1
% http://coursera.org/course/latex


\documentclass[a4paper,12pt]{article}

\usepackage{cmap}
\usepackage[T2A]{fontenc}
\usepackage[utf8]{inputenc}
\usepackage[english,russian]{babel}
\usepackage{fancyhdr}
\usepackage{minted}
\usepackage{hyperref}
\usepackage{amsmath}
\usepackage{graphicx}
\usepackage{xcolor}

\hypersetup{
  pdfborderstyle={/S/U/W 1}
}

\graphicspath{{./images/}}

\pagestyle{fancy}
\fancyhf{}
\lhead{Антон Завьялов, ПИ-72}
\rhead{\textbf{Лабораторная №5}}
\cfoot{\thepage}

\makeatletter
\def\@seccntformat#1{%
  \expandafter\ifx\csname c@#1\endcsname\c@section\else
  \csname the#1\endcsname\quad
  \fi}
\makeatother

\begin{document} % Конец преамбулы, начало текста.

\begin{center}
  \textbf{Лабораторная работа №5 по дисциплине\linebreak"Компьютерная графика"\linebreak\linebreakВыполнил студент группы ПИ-72 Завьялов А.А.}\\
\end{center}

\section{\normalsize{Задание}}
\begin{flushleft}
Для произвольного фотоизображения реализовать:

\begin{enumerate}
    \item вывод изображения на экран;
    \item построение гистограммы яркости;
    \item преобразование яркости;
    \item преобразование контрастности;
    \item изменение цветности: \begin{itemize}
        \item бинаризация (переход к чёрно-белому);
        \item переход к оттенкам серого;
        \item получение негатива;
    \end{itemize}
\end{enumerate}

Дополнительно оцениваются:

\begin{itemize}
    \item работа с выделенным фрагментом;
    \item преобразование гистограмм;
    \item бинаризация с различными методами выбора уровня.
\end{itemize}

\end{flushleft}

\section{\normalsize{Исходный код}}
Исходный код программы также расположен в Git-репозитории по адресу: \url{https://github.com/andiogenes/cg/tree/master/image-processing}
\inputminted[breaklines]{py}{../simplest.py}

\section{\normalsize{Скриншоты, демонстрирующие работу программы}}
\begin{flushleft}
  \includegraphics[scale=0.9]{hist.jpg}
  \includegraphics[scale=1]{lum_1.jpg}
  \includegraphics[scale=1]{lum_2.jpg}
  \includegraphics[scale=1]{contr_1.jpg}
  \includegraphics[scale=1]{contr_2.jpg}
  \includegraphics[scale=1]{binarysation_50.jpg}
  \includegraphics[scale=1]{binarysation_otsu.jpg}
  \includegraphics[scale=1]{grayscale.jpg}
  \includegraphics[scale=0.8]{hist_equal.jpg}
  \includegraphics[scale=1]{negative.jpg}
\end{flushleft}
\end{document}

